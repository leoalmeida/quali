\section{Análise da Humidade}
%http://www.newtoncbraga.com.br/index.php/artigos/54-dicas/2261-art352.pdf
O sensor de umidade é um dispositivo que faz a medição da umidade relativa de uma determinada área, pode ser utilizado tanto para ambientes internos quanto externos. Esses sensores podem ser encontrados tanto em dispositivos analógicos quanto digitais.

\subsection{Ar}

Analógico 
Um sensor de umidade analógico marca a umidade relativa do ar utilizando um sistema capacitivo que são os mais utilizados. Esse tipo de sensor é revestido geralmente de vidro ou cerâmica. O material isolante, que absorve toda a água, é feito de um polímero que recebe e solta a água através da umidade relativa de uma determinada área. Isso modifica o nível de carga presente no capacitor da placa de circuito elétrico.

Digital
O funcionamento de um sensor de umidade digital se dá através de dois micros sensores que são calibrados com a umidade relativa de uma área. Eles são convertidos em um formato digital por um processo de conversão analógico para digital, realizado por um chip localizado no mesmo circuito. 

Uma máquina com um sistema de eletrodos feitos de polímeros é o que produz a capacitância do sensor, que protege o sensor do visor,  que é a interface.

\subsection{Solo}

O sensor de umidade do Solo foi desenvolvido para controlar a influência que a umidade pode provocar em ambientes que ficam expostos às suas influências, haja visto que a umidade do solo é geralmente muito variável, exigindo um controle constante nas medições neste meio.

Esses tipos de sensores servem principalmente para detecção das variações na umidade do solo, sendo aplicado também para uso na terra, areia ou diretamente na água,  senso essencial para controle de irrigações de culturas agrícolas ou irrigação de jardins. Possui um baixo custo e baixo consumo de energia.