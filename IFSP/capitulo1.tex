\chapter{Contextos envolvidos}

\section{Computa\c{c}\~{a}o ub\'{i}qua �  SIoT}
	\subsection{Computa\c{c}\~{a}o Ub\'{i}qua}
	A computa\c{c}\~{a}o ubiqua deu nome � terceira onda da computa\c{c}\~{a}o, iniciada por Mark Weiser em seu artigo ~\cite{Weiser1991}. Segundo outro artigo de sua autoria, ele~\cite{Weiser:1997:CAC:504928.504934} define a terceira onda da computa\c{c}\~{a}o assim; 
	"...Esse terceiro momento ocorre quando as tecnologias recuam para o segundo plano de nossas vidas e muitos computadores (n\~{a}o apenas um) de forma conjunta passam a exercer um papel auxiliar para as pessoas sem tomar tempo do foco principal de sua aten\c{c}\~{a}o.
	
	O principal objetivo das pesquisas na \'{a}rea da computa\c{c}\~{a}o ub\'{i}qua s\~{a}o relacionadas ao refinamento dos dispositivos eletr�nicos de forma que sua utiliza\c{c}\~{a}o seja efetiva e eficiente na cria\c{c}\~{a}o de um contexto computacional~\cite{VanSyckel:2014:SPP:2638728.2641672}, seria como criar uma "conci�ncia virtual" para esse dispositivos de forma que sua utiliza\c{c}\~{a}o possa se tornar transparente para as pessoais. 
	
	Infelizmente, para grande parte das aplica\c{c}\~{o}es nesse paradigma, operar seus dispositivos inteligentes de forma transparente �s pessoas significa ter acesso a informa\c{c}\~{o}es altamente sens\'{i}veis de individuos, exigindo que esses dispositivos sejam projetados cuidadosamente de forma a n\~{a}o se tornar um sistema de fiscaliza\c{c}\~{a}o. Por outro lado, sistemas ub\'{i}quos precisam ser fortemente examinados sobre uma perspectiva de ataques cibern\'{e}ticos, pois j\'{a} se sabe de antem\~{a}o o qu\~{a}o valiosa s\~{a}o suas infoma\c{c}\~{o}es, suscitando grande interesse \'{a} pessoas mal-intencionadas.
	
	\subsection{Disappearing Computer}
	Baseado nas defini\c{c}\~{o}es de Weiser sobre Computa\c{c}\~{a}o Ub\'{i}qua ~\cite{Weiser1991}, surgiu a necessidade de ampliar as pesquisas em tecnologias que fazem dos computadores objetos impercept\'{i}veis, ficando apenas em segundo plano sem tirar o foco das pessoas. A partir dessa disruptura nasceu o termo "Disappearing Computer" com a distin\c{c}\~{a}o dos dispositivos invis\'{i}veis em 2 tipos Desaparecimento F\'{i}sico e Desaparecimento Mental como descrito por Streitz em ~\cite{streitz2001augmented}. 
	
	\paragraph{Desaparecimento F\'{i}sico}
	Como o nome j\'{a} diz, um dispositivo computacional desaparece f\'{i}sicamente, quando ele \'{e} t\~{a}o pequeno que cabe na palma da m\~{a}o, podendo ser costurado em tecidos, colocado junto ao corpo das pessoas ou mesmo implantado no corpo. Na maior parte dos casos, o dispositivo \'{e} integrado � um produto de pequena dimens\~{a}o onde seus recursos computacionais se tornam invis\'{i}veis.
	
	\paragraph{Desaparecimento Mental}
	O aspecto essencial desses cen\'{a}rios \'{e} projetar os dispositivos de forma a n\~{a}o serem mais percebidos como computadores e sim como mais um dos elementos que comp\~{o}e aquele ambiente. Aqui o desaparecimento se limita aos olhos de seus usu\'{a}rios. O dispositivo n\~{a}o precisa ser invis\'{i}vel fisicamente, ele pode estar embutido em portas, janelas e mobili\'{a}rios ou ter apenas sua apar�ncia modificada, dando naturalidade ao inclu\'{i}-lo no ambiente com sua nova roupagem e ficando oculto ao olhar humano.
	
	\subsection{Internet das Coisas - IoT}
	Prop\~{o}e um ecossistema de "coisas" interconectadas. Se entende por "coisa", objetos geralmente compostos por pequenos sistemas computacionais embarcados que contam com a habilidade de captar dados do ambiente em an\'{a}lise trocando informa\c{c}\~{a}o com outras "coisas" diretamente ou atrav\'{e}s de uma rede.
	
	Sua principal caracter\'{i}stica que acaba inviabilizando uma implanta\c{c}\~{a}o universal da Internet das Coisas nos dias atuais \'{e} quanto a forma de busca e comunica\c{c}\~{a}o entre os diversos dispositivos ub\'{i}quos dispon\'{i}veis na rede, que deve ser baseado em um identificador \'{u}nico dentro na rede (ex: Seu IP) ou seja, cada "coisa" conectada � rede ter\'{a} que possuir seu identificador pr\'{o}prio. Por\'{e}m, enquanto a migra\c{c}\~{a}o do protocolo IPv4 para o IPv6 n\~{a}o se concretizar, ser\'{a} econ�micamente invi\'{a}vel implementar um ecossistema novo para a Internet das Coisas que seja amplamente universal. Fora isso, como essa caracter\'{i}stica n\~{a}o \'{e} a \'{u}nica, e novas tecnologias continuam sendo pesquisadas e lan\c{c}adas no mercado, nada impede que sejam feitas adapta\c{c}\~{o}es nas redes existentes para alcan\c{c}ar implementa\c{c}\~{o}es baseadas na Internet das Coisas. A mais promissora delas \'{e} a "Web das Coisas" cujo ecossistema \'{e} baseado em protocolos da Web existentes e que j\'{a} est\~{a}o consolidados, como detalhado no pr\'{o}ximo item.
	
	\subsection{Web das Coisas - WoT}
	A partir de um consenso onde a rede mundial de computadores seria a mais vi\'{a}vel para criar uma Internet das Coisas universalizada, apareceram diversas pesquisas com esse objetivo e continuam aparecendo. Por\'{e}m, a ades\~{a}o ao novo protocolo de roteamento (IPv6) com identificadores \'{u}nicos para todos os dispositivo, primordial para sua consolida\c{c}\~{a}o, n\~{a}o est\'{a} ocorrendo com a velocidado prevista criando um hiato na universaliza\c{c}\~{a}o da internet das coisas, j\'{a} que o protocolo atual (IPv4) n\~{a}o \'{e} adequado a essa nova realidade por sua escassez de identificadores.
	
	Como esse processo de migra\c{c}\~{a}o ser\'{a} lento, pesquisadores e empresas iniciaram suas pesquisas utilizando as estruturas da Web atual, adaptando formas de identificar os elementos dentro da rede. O maior avan\c{c}o na dire\c{c}\~{a}o de criar um elo entre os mundos f\'{i}sicos e virtuais veio da ramifica\c{c}\~{a}o de IoT chamada de "Web das Coisas", que utiliza tecnologias da Web (web services, web sockets e etc.) na identifica\c{c}\~{a}o e comunica\c{c}\~{a}o entre os elementos de uma mesma rede de sensores ub\'{i}quos. Dessa forma, a proposta seria que a manuten\c{c}\~{a}o das redes baseadas na "Web das Coisas" n\~{a}o seja afetada com o fim da migra\c{c}\~{a}o para o protocolo IPv6, havendo uma conviv�ncia harm�nica entre as novas redes de dispositivos e que as novas tecnologias criadas para a Web das Coisas possam ser migradas sem traumas para a Internet das Coisas.
	
	Ainda que a causa principal dessa ramifica\c{c}\~{a}o seja o avan\c{c}o na aplica\c{c}\~{a}o das novas tecnologias, ao avaliar t\'{e}cnicamente a comunica\c{c}\~{a}o entre os diversos dispositivos inteligentes existentes atualmente, vemos que h\'{a} uma grande heterog�neidade de implementa\c{c}\~{o}es, resultando em uma complexidade excessiva para firmar essa comunica\c{c}\~{a}o. Em um futuro pr\'{o}ximo, a falta de um protocolo unificado para troca de mensagens inviabilizaria uma intera\c{c}\~{a}o transparente entre esses dispositivos inteligentes. Na solu\c{c}\~{a}o desse problema, a incorpora\c{c}\~{a}o de mecanismos utilizados na cria\c{c}\~{a}o dos Web Services pela internet tem sido estudada para viabilizar a cria\c{c}\~{a}o de uma conex\~{a}o espont�nea entre os dispositivos inteligentes, apenas fazendo adequa\c{c}\~{o}es necess\'{a}rias � nova realidade da Web das Coisas.
	
	%Inclus\~{a}o de Web Sockets e outras tecnologias web
	
	\subsection{Web Social das Coisas - SWoT}
	Esse paradigma, o mais embrion\'{a}rio dentre os que possuem como origem a Computa\c{c}\~{a}o Ub\'{i}qua, real\c{c}a as pesquisas no campo da intelig�ncia computacional e o aspecto social na comunica\c{c}\~{a}o entre usu\'{a}rios, incluindo dispositivos ub\'{i}quos no relacionamento. Sugundo ~\cite{Mashal2015}, esse relacionamento pode ser atingido com a utiliza\c{c}\~{a}o da Web Sem�ntica e tradu\c{c}\~{a}o de dados estruturados em linguagem natural. Por\'{e}m, precisam ser prospostas novas formas dos dispositivos ub\'{i}quos interpretarem as redes de relacionamento existentes criando uma comunica\c{c}\~{a}o transparente entre dispositivos e se comunicando de forma natural e n\~{a}o intrusiva de forma que eles sejam devidamente compreendidos pelas pessoas.
	

\section{Edge computing e o surgimento da Fog computing}
\subsection{sub-item x}

\section{Cloud of Things and Social Cloud}
\subsection{sub-item x}

\section{Social Virtual Objects}
\subsection{sub-item x}

\section{Web Sem�ntica e Ontologias de IoT}
\subsection{sub-item x}

\section{Mundo de IoT e Minera\c{c}\~{a}o de dados}
Mesmo que pesquisas nesse assunto n\~{a}o tenha rela\c{c}\~{a}o direta � Internet das Coisas, tratar dados coletados por dispositivos ub\'{i}quos possui alta relev�ncia na cria\c{c}\~{a}o de uma Intelig�ncia Computacional. 

Analisar e processar quantidade elevada de dados exige cuidados para evitar que seja extra\'{i}da informa\c{c}\~{a}o sem utilidade ou inconsistente, podendo criar s\'{e}rios problemas. Em projetos de Internet das Coisas esses pontos devem ser observados ao analisar os dados: 

\begin{labeling}{Complexidade}
	\item[Volume]Organiza\c{c}\~{o}es coletam dados de uma grande variedade de fontes, incluindo transa\c{c}\~{o}es comerciais, redes sociais e informa\c{c}\~{o}es de sensores ou dados transmitidos de m\'{a}quina a m\'{a}quina. No passado, armazenar tamanha quantidade de informa\c{c}\~{o}es teria sido um problema ? mas novas tecnologias (como o Hadoop) t�m aliviado a carga.
	\item[Velocidade] Os dados fluem em uma velocidade sem precedentes e devem ser tratados em tempo h\'{a}bil. Tags de RFID, sensores, celulares e contadores inteligentes est\~{a}o impulsionado a necessidade de lidar com imensas quantidades de dados em tempo real, ou quase real. 
	\item[Variedade] Os dados s\~{a}o gerados em todos os tipos de formatos - de dados estruturados, dados num\'{e}ricos em bancos de dados tradicionais, at\'{e} documentos de texto n\~{a}o estruturados, e-mail, v\'{i}deo, \'{a}udio, dados de cota\c{c}\~{o}es da bolsa e transa\c{c}\~{o}es financeiras.
	\item[Variabilidade] Al\'{e}m da velocidade e variedade de dados cada vez maiores, os fluxos de dados podem ser altamente incosistentes com picos peri\'{o}dicos. Existe algo em tend�ncia nas redes sociais? Diariamente, picos de dados sazionais ou picos gerados com base em eventos podem ser um desafio de gerenciar. Ainda mais quando falamos de dados n\~{a}o estruturados. 
	\item[Complexidade] Os dados de hoje vem de v\'{a}rias fontes, o que torna dif\'{i}cil estabelecer uma rela\c{c}\~{a}o, corresponder, limpar e transformar dados entre diferentes sistemas. No entanto, para que seus dados n\~{a}o saiam rapidamente de controle, \'{e} necess\'{a}rio ligar e correlacionar rela\c{c}\~{o}es, hierarquias e as v\'{a}rias liga\c{c}\~{o}es de dados.
\end{labeling}

%verificar se o volume de dados coletados exigir\'{a} cuidados especiais na manuten\c{c}\~{a}o da efici�ncia de processamento
%quais s\~{a}o as origens que os dados poder\~{a}o vir e como identific\'{a}-los, 
%quais metadados a informa\c{c}\~{a}o possui e qual j\'{a} s\~{a}o conhecidos, 
%qual tipo de informa\c{c}\~{a}o pretendemos extrair durante o procesamento dos dados, onde ser\'{a} persistido a informa\c{c}\~{a}o extra\'{i}da nos dados considerando os modelos de apresenta\c{c}\~{a}o dos dados.
%~\cite{TSAI2014}

