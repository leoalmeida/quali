\section{Identifica\c{c}\~{a}o dos recursos na rede}
	Embora seja relativamente novo, o dom\'{i}nio de Internet das Coisas tem sido alvo de pesquisas h\'{a} algum tempo. Universidades, empresas e organiza\c{c}\~{o}es t\^{e}m empregado esfor\c{c}os para definir, propor e implementar solu\c{c}\~{o}es que prover\~{a}o suporte para essa nova \'{a}rea que tende a se popularizar nos pr\'{o}ximos anos.	
	
\subsection{Agrupamento de recursos Federa\c{c}\~{o}es}
	 A infraestrutura baseada em federa\c{c}\~{a}o (IOT-A, 2011) consiste em uma arquitetura proposta com base em conceitos apresentados em (MCLEOD e HEIMBIGNER, 1980) e busca tratar a heterogeneidade de cena?rios e recursos sem exigir uma abordagem que force uma solu\c{c}\~{a}o u?nica para a resolu\c{c}\~{a}o de nomes. Nessa arquitetura, propo?e-se que cada n\'{o} represente um local que agregue diversos recursos e	

\subsection{Agrupamento utilizando RNS}
	Algumas abordagens, como o RNS, proposto em (TIAN et al, 20012), buscam manter a compatibilidade com os sistemas de Internet das Coisas j\'{a} existentes. Projetado para ser uma plataforma capaz de suportar sistemas de nomea\c{c}\~{a}o e resolu\c{c}\~{a}o distintos, o RNS busca n\~{a}o exigir altera\c{c}\~{o}es significativas nos sistemas j\'{a} criados.
