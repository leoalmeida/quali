\section{Redes de Sensores Sem Fio}
	\cite{Silva2009}
	Com a queda crescente nos custos dos equipamentos eletr�nicos, a redu��o no tamanho dos dispositivos simplificando a sua mobilidade, fez com que as redes sem fio ganhassem popularidade rapidamente mundo afora, ganhando novos tipos de aplica��es com objetivo de simplificar o dia-a-dia das pessoas, como previsto por Weiser em \cite{Weiser1991}. 

	Atualmente, a variedade de sensores\cite{SensoresXXXX} existentes e que podem ser obtidos sem muita dificuldade pela internet tem crescido exponencialmente. Aliado � uma consci�ncia dos benef�cios que a interliga��o desses dispositivos em rede para an�lise em tempo real das informa��es ambientais, geralmente invis�veis a nossa aten��o no curto prazo \cite{Weiser:1997:CAC:504928.504934} podem oferecer ao dia-a-dia, faz da rede de sensores sem fio um dos pilares de sustenta��o da Internet das Coisas (IoT).

	Decis�es sobre a arquitetura, modelo e equipamentos utilizados em uma Redes de Sensores Sem Fio \cite{LECKER2010} (RSSF ou WSN) se tornaram fator de sucesso para projetos de IoT.
