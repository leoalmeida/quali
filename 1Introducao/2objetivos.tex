\section{Objetivos}

Como em diversos cen\'{a}rios urbanos da Internet das Coisas, tamb\'{e}m tem se tornado bastante comum em \'{a}reas rurais a utiliza\c{c}\~{a}o de diversos tipos de sensores integrados para compor informa\c{c}\~{o}es mais complexas e relevantes ao neg�cio agr\'{a}rio.

Com isso, considerando os desafios atuais no meio rural e a massifica\c{c}\~{a}o dos equipamentos eletr\^{o}nicos e das redes de Sensores Sem Fio (RSSF), reduzindo de forma substancial seus custos de produ\c{c}\~{a}o.

Dentro dos diversos cen\'{a}rios que podem ser apoiados com a utiliza\c{c}\~{a}o de plataformas IoT, observa-se que aspectos referentes \'{a} identifica\c{c}\~{a}o e ao endere\c{c}amento de recursos se tornam relevantes. Ao mesmo tempo, por possu\'{i}rem especificidades em seus dom\'{i}nios, os mecanismos de nomea\c{c}\~{a}o e acesso a recursos normalmente diferem em cada cen\'{a}rio, tornando mais dif\'{i}cil a integra\c{c}\~{a}o entre os sistemas presentes em cada um deles.

Portanto, os objetivos deste trabalho s�\~{a}o:

\begin{itemize}
	\item Analisar principais recursos naturais, estrat\'{e}gias de armazenamento, sistemas de irriga\c{c}\~{a}o ou distribui\c{c}\~{a}o dos recursos naturais/minerais utilizados na atividade  agropastoril, identificando os atributos relevantes para composi\c{c}\~{a}o de objetos virtuais complexos.
	\item Definir um conjunto de regras para a classifica\c{c}\~{a}o e visualiza\c{c}\~{a}o dos dados recebidos dos recursos conectados \`{a} rede (geradores).
	\item Utiliza\c{c}\~{a}o de uma estrutura na forma de grafo para defini\c{c}\~{a}o de associa\c{c}\~{o}es entre os diversos recursos conectados.
	\item Aplicar t\'{e}cnicas de nomea\c{c}\~{a}o e recomenda\c{c}\~{a}o para otimiza\c{c}\~{a}o da busca dos recursos conectados \`{a} rede, de modo que sejam selecionados apenas os recursos mais adequados \`{a}s regras definidas nos canais de publica\c{c}\~{a}o.
	\item Aplicar metodologias de ICVs e ACVs na an\'{a}lise dos custos de avalia\c{c}\~{a}o do desempenho ambiental da \'{a}rea avaliada.
	\item Elaborar uma plataforma distrib\'{i}da de apoio ao processo de mapeamento das regras de publica\c{c}\~{a}o que ser�o utilizados em canais complexos, an\'{a}lise e visualiza\c{c}\~{a}o da informa\c{c}\~{a}o sendo processada em tempo real, considerando os diversos aspectos que influenciem esse processamento, como as caracter\'{i}sticas do conjunto de dados e informa\c{c}\~{o}es obtidas de bases oficiais (ICVs e ACVs).
\end{itemize}
