\chapter{Cen�rios}
\section{Cidades inteligentes e IoT}15pg
\subsection{Aplica��es}
\subsection{Desafios}
Tamb�m os sistemas evolutivos passam a ter, atualmente, maior relev�ncia no tratamento do problema da customiza��o e est�o sendo testados objetivando as ind�strias inteligentes e as exig�ncias da Ind�stria 4.0. Mais recentemente surgiu um novo paradigma denominado de \textit{SAS (Symbiotic Assembly System)}~\cite{FERREIRA2014} que procura integrar os sistemas de automa��o r�gido, flex�vel e program�vel �s capacidades do ser humano. % que neste paradigma faz parte ativa da modelagem.
