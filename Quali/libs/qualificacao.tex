%% Todas as opcoes da classe abnt (AbnTeX) sao validas.
%% Outras opcoes sao: quali e tese (dissertacao e' padrao)
\documentclass[quali]{ppgccufscar}
%\documentclass[]{abntex2}
% \usepackage{lmodern}			% Usa a fonte Latin Modern	

%% pacotes incompativeis:
%%   qualquer pacote de citacoes, como natbib, apalike, cite, etc.
%% o pacote babel ja vem carregado com ingles e portugues
%\usepackage[brazil]{babel}
%\usepackage[utf8]{inputenc}
\usepackage[latin1]{inputenc}
\usepackage[T1]{fontenc}

\usepackage{upgreek}
\usepackage{mathrsfs}
\usepackage{amsmath}
\usepackage{graphicx}
\usepackage{graphics}

\usepackage{enumerate}
\usepackage{multirow}
\usepackage{verbatim}

\usepackage{indentfirst}% Indenta o primeiro par�grafo de cada se��o.

% Color control.
\usepackage[usenames,dvipsnames]{xcolor}

% Soul package: to use the \hl{} command to highlight text, useful to spot "to-do" items.
% The command \phl is defined so a different highlight color is used by the professor in his/her notes.
\usepackage{soul}
%\usepackage{soulutf8}
\newcommand{\phl}[2][Peach]{{\sethlcolor{#1} \hl{#2}}}

%%%%%%%%%%%%%%%%%%%%%%%%%%%%%%%%%%%%%%%%%%%%%%%%%%%%%%%%%%%%%%%%%%%%%%%%%%%%% 
% Definitions

\graphicspath{{../../images/}}
%\bibliographypath{{../../references/}}



\newtheorem{mydef}{Defini\c{c}\~ao}

\newenvironment{bigequation}{\begin{equation} \fontsize{16pt}{20pt}}{\end{equation}}

\newcommand{\cc}[2][c]{%
	\begin{tabular}[#1]{@{}c@{}}#2\end{tabular}}

\title{Quali leo}

%%%%%%%%%%%%%%%%%%%%%%%%%%%%%%%%%%%%%%%%%%%%%%%%%%%%%%%%%%%%%%%%%%%%%%%%
% Altere os campos abaixo para alterar o documento final
%%%% T�tulo da Pesquisa
\titulo { Tema principal }
%%%%
\autor{Leonardo Almeida Silva Ferreira}
%%%%
\orientador[Orientador]{Prof. Dr. Helio Cristana Guardia}
%%%%
%\coorientador{Prof. Dr. XXXX}

%%%% �rea de Concentra��o da pesquisa
\areaconcentracao{Sistemas Distribu\'idos}

%%%% Data localizada no rodap� da p�gina
\data{Abril/2017}

%%%% Nome da institui��o (UFSCAR � default)
%\instituicao{}

%%%% Local localizado no rodap� da p�gina (S�o Carlos � default)
%\local{}

%%%% Coment�rio que fica na 2a pag. em baixo do nome � direita
\comentario{\PPGtipodoc\ apresentada ao Programa de P\'os-Gradua\c{c}\~ao em Ci\^encia da Computa\c{c}\~ao da Universidade Federal de S\~ao Carlos, como parte dos requisitos para a obten\c{c}\~ao do t\'{\i}tulo de \PPGtipotitulo\ em Ci\^encia da Computa\c{c}\~ao, \'area de concentra\c{c}\~ao: \PPGareaconcentracao}

%%%% Remove ou altera os Detalhes que ficam embaixo do nome da institui��o
%\instituicaodetails{}
%%%%%%%%%%%%%%%%%%%%%%%%%%%%%%%%%%%%%%%%%%%%%%%%%%%%%%%%%%%%%%%%%%%%%%%%

% epigrafre, agradecimentos sao feitos 'na mao'.
% um dia eu faco alguns comandos para eles ;)

\begin{document} 
	
\capa
\folhaderosto
%\folhadeaprovacao
%\endfolhadeaprovacao

%\dedicatoria{A meus pais.}
%\begin{agradecimentos}
Agradeço ao \LaTeX por não ter vírus de macro.
Agradeço ao Linus Torvalds por ter feito o Linux.
Agradeço ao Bill Gates por deixar-nos piratear seus softwares.
\end{agradecimentos}

%\epigrafe{Um dia.}{Dadavski Shoffstall}

\begin{resumo}
	Nonono nonono nonono, nonono, nonono nonono nonono nononono nonno.
	Nonono nonono nonono, nonono, nonono nonono nonono nononono nonno.
	Nonono nonono nonono, nonono, nonono nonono nonono nononono nonno.
	Nonono nonono nonono, nonono, nonono nonono nonono nononono nonno.
	Nonono nonono nonono, nonono, nonono nonono nonono nononono nonno.
	Nonono nonono nonono, nonono, nonono nonono nonono nononono nonno.
	Nonono nonono nonono, nonono, nonono nonono nonono nononono nonno.


\palavraschave{
	Computa\c{c}\~{a}o M\'{o}vel,  
	Comunica\c{c}\~{a}o sem Fio, 
	Comunica\c{c}\~{a}o Oportun\'{i} stica }
\end{resumo}

\begin{abstract}
Nonono nonono nonono, nonono, nonono nonono nonono nononono nonno.
Nonono nonono nonono, nonono, nonono nonono nonono nononono nonno.
Nonono nonono nonono, nonono, nonono nonono nonono nononono nonno.
Nonono nonono nonono, nonono, nonono nonono nonono nononono nonno.
Nonono nonono nonono, nonono, nonono nonono nonono nononono nonno.
Nonono nonono nonono, nonono, nonono nonono nonono nononono nonno.
Nonono nonono nonono, nonono, nonono nonono nonono nononono nonno.

\keywords{	ph.d. dissertation,
			dissertation,
			monograph,
			project}
\end{abstract}


\listoffigures
\listoftables

%% sumario
\tableofcontents

%%%%%%%%%%%%%%%%%%%%%%%%%%%%%%%%%%%%%%%%%%%%%%%%%%%%%%%%%%%%%%%%%%%%%%%%
%% aqui comeca o texto da disserta��o
\chapter{Introdu\c{c}\~ao/motiva\c{c}\~ao}
No in�cio da transmiss�o um n� � definido como fonte. Ao entrar um n� destino (receptor da transmiss�o), o algoritmo de Floyd-Warshall � utilizado para dividir o fluxo, levando em considera��o a capacidade de cada aresta. Quando outro n� destino entrar, uma busca em largura � realizada para encontrar as partes mais pr�ximas que completam o fluxo e estas s�o transmitidas para o novo host, por um caminho determinado pelo algoritmo Floyd-Warshall. Um exemplo pode ser observado na figura 
\begin{figure}
	\centering
	\includegraphics[scale=0.6]{Grafo_exemplo_mestrado.png}
	\caption{"Grafo exemplo de topologia"}
	\label{fig:topo_example}
\end{figure}

\section{Contextualiza��o}
\subsection{Linux}
\section{Motiva��o e Objetivos}
\subsection{Linux}
\section{Metodologia}
\subsection{Linux}
\section{Estrutura��o do texto}
\subsection{Linux}
Utilize o BibTeX para organizar as suas refer�ncias. 
 %Introdu��o

\chapter{Trabalhos Relacionados}
\section{X}
\subsection{Sub X} %Cen�rios - Cidades inteligentes e IoT

\chapter{Trabalhos Relacionados} 5pg
\section{Temas t�cnicos relacionados ao trabalho}
\subsection{Aplica��es} %Trabalhos Relacionados - Temas técnicos relacionados ao trabalho

\chapter{Proposta de trabalho que será desenvolvida}
\section{Outros aspectos que impactam a parte principal descrita anteriormente} 5pg
\subsection{sub-item x} %Proposta de trabalho desenvolvida

\input{../capitulo5-conclusao} %Conclus�o


\bibliography{../../references/central-bibliography} % bibname=nome do seu arquivo BibTeX
%%
%%%%%%%%%%%%%%%%%%%%%%%%%%%%%%%%%%%%%%%%%%%%%%%%%%%%%%%%%%%%%%%%%%%%%%%%
%% DICA: voce pode ir definindo os acronimos ao longo do texto.
%% Por exemplo, no capitulo 1, vc ta escrevendo:
%% Segundo Fulano, Model-Driven Development (MDD)\acronym{MDD}{Model-Driven Development} ? uma t?cnica bla bla bla...
%\acronym{MDD}{Model-Driven Development}
\listofacronyms

\apendice

\anexo
	
\end{document}