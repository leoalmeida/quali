\section{An�lise da temperatura}
%http://www.resistorguide.com/photoresistor/
Sensores de temperatura ou Termistores servem para indicar a temperatura do ambiente que est� sendo controlado. Sua medi��o � feita mensurando as altera��es de resistencia �  alterada termicamente, isto �, apresentam um valor de resist�ncia el�trica para cada temperatura absoluta.

Thermistors are resistors of which the resistance changes significantly when temperature changes. Different thermistor types exist, the two most common types are the NTC and PTC thermistor. NTC thermistors decrease in resistance when the temperature rises, while PTC thermistors increase in resistance when the temperature rises. Thermistors are often used as temperature sensors or thermal protection devices.

 They are primarily used as resistive temperature sensors and current-limiting devices. 

\begin{itemize}
	\item{RTD} feitos com uso de metais
	\item{PTC} S�o tipos de resistores cujo coeficiente de temperatura � positivo, significando que a resistencia aumenta quando h� aumento da temperatura.
	\item{NTC} S�o tipos de resistores cujo coeficiente de temperatura � negativo, significando que a resistencia diminui quando h� aumento da temperatura. Diferente dos sensores RTD, esses componentes s�o geralmente fabricados em cer�mica ou pol�meros.
\end{itemize}

Como o meterial de fabrica��o e outras caracter�sticas dos sensores de temperatura influ�nciam no resultado de sa�da, essas informa��es possuem grande relev�ncia na convers�o dos dados de sa�da para a ontologia utilizada.
 
 Seu coeficiente de sensitividade t�rmica � maior que os sensores de temperatura feitos com silicio (PTC) e dos sensores resistivos (RTD), podendo ser usados para leituras de temperatura que variam entre ?55�C e 200�C.

\subsection{Compara��o entre os tipos de sensores de temperatura}

Compared to RTDs, the NTCs have a smaller size, faster response, greater resistance to shock and vibration at a lower cost. They are slightly less precise than RTDs. When compared to thermocouples, the precision obtained from both is similar; however thermocouples can withstand very high temperatures (in the order of 600�C) and are used in such applications instead of NTC thermistors, where they are sometimes referred to as pyrometers. Even so, NTC thermistors provide greater sensitivity, stability and accuracy than thermocouples at lower temperatures and are used with less additional circuitry and therefore at a lower total cost. The cost is additionally lowered by the lack of need for signal conditioning circuits (amplifiers, level translators, etc.) that are often needed when dealing with RTDs and always needed for thermocouples.


\acronym{NTC}{Negative Temperature Coefficient}
\acronym{PTC}{Positive Temperature Coefficient}
\acronym{RTD}{Resistance Temperature Detectors}
