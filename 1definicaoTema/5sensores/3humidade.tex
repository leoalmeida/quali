\section{An�lise da Humidade}
%http://www.newtoncbraga.com.br/index.php/artigos/54-dicas/2261-art352.pdf
O sensor de umidade � um dispositivo que faz a medi��o da umidade relativa de uma determinada �rea, pode ser utilizado tanto para ambientes internos quanto externos. Esses sensores podem ser encontrados tanto em dispositivos anal�gicos quanto digitais.

\subsection{Ar}

Anal�gico 
Um sensor de umidade anal�gico marca a umidade relativa do ar utilizando um sistema capacitivo que s�o os mais utilizados. Esse tipo de sensor � revestido geralmente de vidro ou cer�mica. O material isolante, que absorve toda a �gua, � feito de um pol�mero que recebe e solta a �gua atrav�s da umidade relativa de uma determinada �rea. Isso modifica o n�vel de carga presente no capacitor da placa de circuito el�trico.

Digital
O funcionamento de um sensor de umidade digital se d� atrav�s de dois micros sensores que s�o calibrados com a umidade relativa de uma �rea. Eles s�o convertidos em um formato digital por um processo de convers�o anal�gico para digital, realizado por um chip localizado no mesmo circuito. 

Uma m�quina com um sistema de eletrodos feitos de pol�meros � o que produz a capacit�ncia do sensor, que protege o sensor do visor,  que � a interface.

\subsection{Solo}

O sensor de umidade do Solo foi desenvolvido para controlar a influ�ncia que a umidade pode provocar em ambientes que ficam expostos �s suas influ�ncias, haja visto que a umidade do solo � geralmente muito vari�vel, exigindo um controle constante nas medi��es neste meio.

Esses tipos de sensores servem principalmente para detec��o das varia��es na umidade do solo, sendo aplicado tamb�m para uso na terra, areia ou diretamente na �gua,  senso essencial para controle de irriga��es de culturas agr�colas ou irriga��o de jardins. Possui um baixo custo e baixo consumo de energia.