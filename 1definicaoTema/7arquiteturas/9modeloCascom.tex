\section{Modelo CASCoM}
%introduction
Em /cite{CASCoM}, Perera propõe um modelo de composição de sensores baseado em contexto com a finalidade de simplificar o processo de configuração de sensores com a aplicação de mecanismos de filtragem, associações e lógica nas plataformas de IoT com a incorporação de tecnologias de semântica para solução dos desafios propostos.
Para isso, ele separa as configurações necessárias ao paradigma de IoT em 2 grandes categorias: configurações em nível de sensores e em nível de sistema. Sendo a primeira categoria focada no controle do comportamento dos sensores e a última na configuração dos parâmetros de software relacionados à atividade dos sensores como: escalonamento de leituras, taxa de amostragem, frequência de comunicação de dados, padrões de comunicação utilizados e protocolos.

De forma mais especifica, o modelo identifica, detalha suas composições e configura a atuação dos sensores e o processamento dos dados coletados de acordo com os objetivos dos usuários.
Este modelo difere em relação ao proposto no modelo GSN ....

In existing IoT middleware (e.g. GSN), many configuration files and programming codes need to be manually defined by the users (without any help from GSN). An ideal IoT middleware configuration model should address all the above mentioned chal- lenges. The configuration model we propose in this paper is applicable towards several other emerging paradigms, such as sensing as a service [4]. 



%model

Seu modelo de solução, foi dividido nas 6 fases seguintes: Understand User Requirements; Select Data Processing Components; Select Sensors; Provide advice and Recommendations; Discover Additional Context; Context-based Cost Optimization.A saída do processo são as aplicações dos dados coletados atendendo às necessárias do usuário como demonstrado pela figura X abaixo.

%SSN - w3.org/2005/Incubator/ssn/wiki/SSN

Para criar a semântica de seu modelo, Perera utilizou a essência de 3 diferentes ontologias (QA-TDO, SCO e SSN) para descrição dos diferentes tipos de sensores, componentes de software e domínios de conhecimento.

In phase 1, users are facilitated with a graphical user interface, which is based on a question-answer (QA) approach, that allows to express the user requirements. Users can answer as many question as possible. CASCoM searches and filters the tasks that the user may want to perform. From the filtered list, users can select the desired task. The details of the QA approach are presented later in this section. In phase 2, CASCoM searches for different programming components that allow to generate the data stream required. In phase 3, CASCoM tries to find the sensors that can be used to produce the inputs required by the selected data processing components. If CASCoM fails to produce the data streams required by the users due to insufficient resources (i.e. unavailability of the sensors), it will provide advice and recommendations on future sensor deployments in phase 4. Phase 5 allows the users to capture additional context information. The additional context information that can be derived using available resources and knowledge are listed to be selected. In phase 6, users are provided with one or more solutions3. CASCoM calculates the costs for each solution in using technique disucced in [2]. By default, CASCoM will select the solution with lowest cost. However, users can select the cost models as they required. Finally, CASCoM generates all the configuration files and program codes which we listed in Figure 2(a). Data starts streaming soon after. 

%Evaluation, Discussion and Lessons Learned

Results: Figure 5(a) shows that CASCoM allows to considerably reduce the time required for configuration of data processing mechanism in IoT middleware. Specifically, CASCoM allowed the three types of users to complete the given task 50, 80 and 250 times faster (respectively) in comparison to the existing approach. According to Figure 5(b), the Java reflection approach takes slightly more time to specially when initializing. Though the Java reflection approach can add more flexibility to our model, the additional overhead increases when the number of components and operation involved gets increased. The overheads can grow up to unacceptable level very quickly when GSN scales up (e.g. more user requests).

According to Figure 5(c), even IT experts who know GSN can save time by using CASCoM up to 88%. Specially, time taken for defining the VSD and VS class have been significantly reduced. Both files can be generated by CASCoM autonomously within a second even for complex scenarios. However, the time taken to find data processing components and sensors (and wrappers) depends on the size of the semantic data model. Figure 5(d) shows how total processing time would vary depending on the size of the semantic data model. Approximately, a semantic model with 10,000 sensor descriptions and 10,000 data processing components can be processed in order to find solutions for a given user request in less than a minute. However, most of the time is taken to read the data model. 

The actual configuration process other than reading the data model takes only

4 seconds and it slightly increases when the model size increases. 

%Conclusion

We have shown that it is possible to offer a sophisticated configuration model to support non-IT experts. Semantic technologies are used extensively to support this model. Using our proof of concept implementation, both IT and non-IT experts were able to configure the GSN in significantly less time. In future, we plan to extend our configuration model into sensor-level. To achieve this, we will develop a model that can be used to configure sensors autonomously without human intervention in highly dynamic smart environments in the IoT paradigm. 
