\section{Plataforma Perera}
%http://ieeexplore.ieee.org/document/6605518/
%\cite{cite1}http://www.sciencedirect.com/science/article/pii/S0950705116302015
%\cite{cite2}http://ieeexplore.ieee.org/document/6468408/citations?anchor=anchor-paper-citations-ieee&ctx=citations
%\cite{cite3}http://ieeexplore.ieee.org/document/7983223/
%http://ieeexplore.ieee.org/document/6569153/
% Os termos objetos, coisas, objetos inteligentes, dispositivos e nós são utilizados nesse trabalho com o mesmo significado, já que assim é utilizado na literatura sobre IoT. 

Em XXXxX, Perera descreve o significado de uma arquitetura de sensores em plataformas de IoT, 
n the sensing as a service paradigm, Internet of Things (IoT) Middleware platforms allow data consumers to retrieve the data they want without knowing the underlying technical details of IoT resources (i.e. sen- sors and data processing components). However, configuring an IoT middleware platform and retrieving data is a significant challenge for data consumers as it requires both technical knowledge and domain expertise. In this paper, we propose a knowledge driven approach called Context Aware Sensor Config- uration Model (CASCOM) to simplify the process of configuring IoT middleware platforms, so the data consumers, specifically non-technical personnel, can easily retrieve the data they required. In this pa- per, we demonstrate how IoT resources can be described using semantics in such away that they can later be used to compose service work-flows. Such automated semantic-knowledge-based IoT resource composition approach advances the current research. We demonstrate the feasibility and the usability of our approach through a prototype implementation based on an IoT middleware called Global Sensor Networks (GSN), though our model can be generalized to any other middleware platform. 

Em seu trabalho, Pereira identificou os principais pontos necessários ao desenvolvimento de um modelo para IoT que viabilize a disponibilização de sensores como serviços através da composição de múltiplos tipos de sensores utilizando diferentes mecanismos de filtros, associações e lógicas. Seu modelo é composto das seguintes funcionalidades principais:

Modelo Autônomo:

Baseado no uso (Utility based):

Escalável e flexível:

Fácil utilização e baixa curva de aprendizado:

