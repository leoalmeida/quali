\section{Qualidade da água Doce}
A qualidade da água pode ser analizada utilizando os índices Demanda Bioquímica de Oxigênio(DBO) e Índice de Qualidade da Água (IQA)

\subsection{Demanda Bioquímica de Oxigênio - DBO} Mede a quantidade de
oxigênio necessária para degradar bioquimicamente a matéria orgânica presente
na água. Quanto maior o seu valor, pior é a qualidade da água.

\subsection{ Índice de Qualidade da Água - IQA} O IQA foi obtido a partir de um estudo realizado, em 1970, pela National Sanitation Foundation - NSF, fundada, em 1944, pela Escola de Saúde Pública da Universidade de Michigan, dos Estados Unidos, tendo sido adaptado e desenvolvido pela Companhia Ambiental do Estado de São Paulo - Cetesb. Para a sua obtenção é necessário aplicar uma fórmula matemática que utiliza nove parâmetros, considerados relevantes para a avaliação da qualidade das águas, tendo como determinante principal a sua utilização para o abastecimento público: temperatura da amostra; pH, oxigênio dissolvido; demanda bioquímica de oxigênio; coliformes termotolerantes; nitrogênio total; fósforo total; resíduo total; e turbidez. No caso de não se dispor do valor de algum desses parâmetros, o cálculo do IQA é inviabilizado.

A partir do cálculo efetuado, pode-se determinar a qualidade das águas brutas, que é indicada pelo IQA, variando numa escala de 0 a 100, conforme o Quadro 2. Quanto maior o valor do IQA, melhor a qualidade da água. Nem todos os órgãos e agências ambientais fazem uso desse índice.

% Please add the following required packages to your document preamble:
% \usepackage{booktabs}
\begin{table}[]
	\centering
	\caption{My caption}
	\label{my-label}
	\begin{tabular}{@{}ll@{}}
		\toprule
		\multicolumn{1}{r}{Categoria} & Faixa de valor         \\ \midrule
		Ótima                         & 79 \textless IQA ≤ 100 \\
		Boa                           & 51 \textless IQA ≤ 79  \\
		Regular                       & 36 \textless IQA ≤ 51  \\
		Ruim                          & 19 \textless IQA ≤ 36  \\
		Péssima                       & IQA ≤ 19               \\ \bottomrule
	\end{tabular}
\end{table}

As comparações entre os resultados de DBO e IQA em diferentes rios devem ser feitas levando em conta que tanto a intensidade temporal e espacial das amostragens quanto os métodos de análise dos parâmetros mensurados variam entre os órgãos ambientais. Além disso, os valores refletem a qualidade das águas no momento da coleta, considerando as variações meteorológicas e as características dos efluentes lançados nos rios pelas indústrias e domicílios. A caracterização da forma de obtenção das informações é apresentada no Quadro 3.


O Conselho Nacional do Meio Ambiente - Conama estabelece cinco classes de água doce, cada uma com valores
de qualidade de água apropriados ao uso predominante, recomendado
para o abastecimento humano, a recreação, a irrigação, a navegação etc. Para
as águas de Classe 2, por exemplo, o Conama estabelece o valor de 5 mg/l como
limite máximo para a DBO, podendo ser usadas no abastecimento público, após
tratamento convencional. Mensurações periódicas nas águas dos rios permitem
aferir se a sua qualidade é apropriada aos usos que lhes são conferidos.

A DBO e o IQA são instrumentos fundamentais para o diagnóstico da
qualidade ambiental de águas interiores, sendo importantes também para o
controle e o gerenciamento dos recursos hídricos, estando entre os indicadores
mais utilizados mundialmente na aferição da poluição hídrica. A DBO é um
parâmetro importante no dimensionamento de uma Estação de Tratamento de
Água - ETA ou Estação de Tratamento de Efluentes - ETE.
Enquanto a DBO evidencia o lançamento de esgotos domésticos na água,
o IQA é um indicador mais genérico, revelador do processo de eutrofização das
águas. Associados a outras informações ambientais e socioeconômicas são
bons indicadores de desenvolvimento sustentável.

