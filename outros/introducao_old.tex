Atualmente estamos vivendo em um mundo onde a utiliza��o de redes sociais por pessoas para aprender, se conectar aos amigos, conhecer novas pessoas, buscar o melhor caminho e at� mesmo facilitar atividades do dia-a-dia est�o consolidadas na rotina de todos. Al�m disso, as pessoas tem buscado formas de viabilizar o uso compartilhado de equipamentos em geral, buscando uma redu��o de espa�o e seus custos de obten��o e manuten��o. 

Por outro lado, para se manter na vanguarda das novas tecnologias, as fabricantes de equipamentos eletro eletr�nicos est�o investindo fortemente para tornar seus produtos m�veis e conectados, fortalecendo assim o contato com seus clientes e buscando conquistar um espa�o nessa era social.

A expans�o desse contexto social tem gerado uma quantidade crescente de informa��es com sem�nticas cada vez mais complexas, onde uma hierarquia de dados s�o utilizadas na cria��o de novos objetos virtuais complexos a partir das ontologias j� existentes. No entanto, atualmente, � not�vel a diversidade de possibilidades para representa��o desses objetos complexos. Essa grande variedade de t�cnicas para mapeamento acaba por dificultar na escolha da t�cnica mais apropriada para a representa��o de um conjunto de objetos no mundo de IoT. 

Desta forma, fica estabelecido um cen�rio onde a escolha da estrat�gia de cria��o dos objetos 

A proposta desta disserta��o � definir uma arquitetura que facilite a sele��o e cria��o de objetos de forma distribu�da, qu
