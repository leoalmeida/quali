\section{Aplica\c{c}\~ao Localizada} 

 O QUE É  IRRIGAÇÃO LOCALIZADA?
           É  um método de irrigação em que a água é aplicada na área ocupada pelas raízes das plantas. O gotejamento e a microaspersão são os sistemas mais conhecidos, muito embora existam outros utilizados em pequenas propriedades, como xiquexique e cápsulas porosas.
 QUAIS AS VANTAGENS DA IRRIGAÇÃO LOCALIZADA?
 ·        Alta eficiência de aplicação de água e fertilizantes;:
·        utilização de bombas de baixa potência;
·        economia de energia;
·        possibilidade de uma jornada de trabalho de 24 horas/dia; e
·        maior  automação  do sistema.
QUAIS OS EQUIPAMENTOS NECESSÁRIOS PARA SE FAZER A IRRIGAÇÃO LOCALIZADA?
·        Sistema de bombeamento;
·        cabeçal de controle;
·        tubulações principal e secundária;
·        válvulas de controle de pressão e vazão;
·        linhas de distribuição de água;
·        gotejadores ou tubos gotejadores; e
·        microaspersores.
 
COMO É FEITA A IRRIGAÇÃO LOCALIZADA?
 A água é aplicada por meio de gotejadores ou microaspersores e distribuída de forma constante, lenta e à baixa pressão, formando uma área úmida em torno das raízes chamada bulbo molhado, típico do gotejamento.
 