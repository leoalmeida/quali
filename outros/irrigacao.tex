\section{Quando deve ser feita a irriga\c{c}\~ao Localizada}

\cite{COELHO2017} A hora de irrigar pode ser decidida de duas formas:
\begin{itemize}
	\item Estabelecer os dias da semana para irrigar;
	\item Se o solo for arenoso, irrigar todo dia, independente de tempo nublado ou com sol;
	\item Se o solo for mais argiloso ou barrento, e se observar que irrigando todo dia a terra fica muito úmida até um dia depois da irrigação, então irrigar dia sim, dia não. 
\end{itemize}

Dar preferência ao início da manhã para irrigar. Verificar a umidade do solo todos os
dias antes da irrigação. Para verificar a umidade do solo, é preciso conhecer como ver se
o solo está na umidade boa para as plantas ou não. Para isso o modo mais fácil é o método
do uso das mãos.


\subsection{Uso de equipamentos na aferição adequada de humidade do solo}
O Irrigas é um equipamento de baixo custo e de fácil manuseio. Determina apenas
o momento de irrigar ou não. Ele é composto por uma cápsula porosa tipo vela de filtro
conectada a uma mangueira fina (micro tubo) com uma cuba de plástico na ponta
da mesma.


\subsection{Medidas para controle e redução da água na irrigação}
	Cobertura no solo: A cobertura do solo na região que é
	molhada pela irrigação pode reduzir bastante o tempo de irrigação porque não deixa a
	água evaporar da terra mantendo mais umidade o tempo todo para as plantas. 

	A quantidade de água a aplicar nas plantas vai depender do consumo dessas plantas.
	Plantas novas consomem menos água que plantas mais adultas em fase de floração e
	enchimento dos frutos. Em dias ensolarados, com ar seco e vento as plantas precisam de
	mais água que emdias úmidossemvento, ou nublados.Se a terra (solo)temcobertura como
	palha seca seránecessáriomenostempode irrigaçãocomparadoaplantas como solonu